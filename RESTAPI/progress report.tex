\documentclass[10pt, a4paper]{beamer}
\usepackage{graphicx}
\usetheme{Berkeley}
\usecolortheme{sidebartab}

\begin{document}
	\setbeamertemplate{sidebar left}{}
	\title{Progress Presentation-I}
	\subtitle{e-Yantra Summer Internship-2019 \\ Auto Tagging Algorithm 2.0}
	\author{Anuj Trehan\\Mentors : Sharad Mishra, Tenzin Dhekyong}
	\institute{IIT Bombay}
	\date{\today}
	%\addtobeamertemplate{sidebar left}{}{\includegraphics[scale = 0.3]{logowithtext.png}}
	\frame{\titlepage}

\setbeamertemplate{sidebar left}[sidebar theme]
\section{Overview of Project}
\begin{frame}{Overview of Project}
	\begin{itemize}
		\item \textbf{Project Name :} Auto Tagging Algorithm 2.0
		\item \textbf{Objective :} Accurately predicting the difficulty levels of question mapped manually or changing it according to features and reading its performance over time.

		\item \textbf{Deliverables :} Documentation and analysis of different techniques and there accuracy in comparison with themselves.
	\end{itemize}
\end{frame}

\section{Overview of Task}
\begin{frame}{Overview of Task}
	\begin{tabular}{|l|l|l|}
		\hline
		S.No & Task & Deadline\\
		\hline
		1 & Reading the research paper & 1 Day \\
		\hline
		2 & Understanding the data and making DB connector & 1 Day  \\
		\hline
		3 & Feature creation from database & 2 Days \\
		\hline
		4 & Implementing unsupervised learning techniques & 8 days \\
		\hline
		5 & Calculate accuracy and performance of the result & 2 days\\
		\hline
		6 & Testing of algo for performance improvement & 3 days\\
		\hline
		7 & Integrating with REST or HTTP API & 1-2 days\\
		\hline
		8 & Making analytics dashboard & 4 days\\
		\hline
		9 & Debugging of Software Stack & 2 days\\
		\hline
		10 & Testing/Documentation & 3 days\\
		\hline
	\end{tabular}
\end{frame}

\section{Task Accomplished}
\begin{frame}{Task Accomplished}
	\begin{itemize}
		\item Read and understood \textit{Auto-Tagging for Massive Online Selection Tests: Machine Learning to the Rescue}.  
		\item DB connector was made in python.
		\item Features were made from the database.
		\item K-Means Clustering algorithm was applied on the features and accuracy was calculated.
	\end{itemize}
\end{frame}

\section{Challenges Faced}
\begin{frame}{Challenges Faced}
	\begin{itemize}
		
		\item Understanding the whole database to find out important tables.There were 10 tables in the database but only \textbf{4} of them were needed.\\ 
		
		\item Making Features from different tables and import it into python.This was the most important step as it forms the base for our model.\textbf{Good data promises good models.}\\ 
		
		\item Calculating the features from research paper was time consuming. Firstly we have to \textbf{calculate weights for each student and also map each question to the students who answered them} and then calculate the features.Shown below are  equations :- \\
		\begin{equation}
		    F_{k}^{(1)} = \frac{\sum_{s\in S} w_{s} f_{k}^{(1)} (D_{k}^{(s)})}{\sum_{s\in S} w_{s}}
		\end{equation}
		\\
		\begin{equation}
		    F_{k}^{(2)} = \frac{\sum_{s\in S} w_{s} m_{s} f_{k}^{(2)} (D_{k}^{(s)})}{\sum_{s\in S} w_{s}}
		\end{equation}
		
	\end{itemize}
	
\end{frame}
\begin{frame}{Challenges Faced (cont...)}
	\begin{itemize}
		\item When K-Means clustering was applied to the features the label assigned to the cluster was changing which changed the whole metric on each run.\\
		\begin{center}
		\includegraphics[width=2cm, height=2cm]{first_run.png}\\
		FIRST RUN
	\end{center}
		\begin{center}
		\includegraphics[width=2cm, height=2cm]{second_run.png}\\
		SECOND RUN
	\end{center}
	\end{itemize}
\end{frame}

\section{Future Plans}
\begin{frame}{Future Plans}
	\begin{itemize}
		\item Applying more learning models to the dataset like Autoencoder pretraining,competitive learning etc.
		\item Bind the model into a REST API and communicate with a frontend.
		\item Making a simple dashboard
	\end{itemize}
\end{frame}


\section{Thank You}
\begin{frame}{Thank You}
	\centering THANK YOU !!!
\end{frame}
\end{document}
